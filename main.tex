% arara: lualatex
% arara: bibtex
% arara: lualatex
% arara: lualatex


% Pattern Languages of Programs Conference 2023 October 22-25, 2023

% Original timeline:

% June 2         Shepherding begins
%% - We'll know if we have a shepherd
% June 30        Deadline focus group / workshop proposals
% August 7       Second draft due for review
% August 15      Notification of acceptance
% September 18   Conference versions due
% October 1      Conference registration ends
% October 22     Patterns Bootcamp
% October 23-25  PLoP Conference Days

%% <-- WE’RE HERE NOW

% January 29, 2024 Proceeding version due

% Created 2021-11-16 Tue 16:22
% Intended LaTeX compiler: xelatex
\RequirePackage[prologue,table,dvipsnames]{xcolor}
\documentclass[acmlarge,timestamp]{acmart}

\providecommand{\tightlist}{%
  \setlength{\itemsep}{0pt}\setlength{\parskip}{0pt}}

\usepackage[many]{tcolorbox}
\usepackage{varwidth}
\usepackage{environ}
\usepackage{xparse}
\usepackage{microtype}
\usepackage{tikz}
\usepackage{calc}

%\usepackage[all]{xy}
%\usepackage{rotating}

\newlength{\somelengthA}
\newlength{\somelengthB}
\newlength{\somelengthC}

\newcommand\clock[3]{%
\begin{tikzpicture}[line cap=round,line width=3pt]
\filldraw [fill=Goldenrod!20,thin] (0,0) circle (8cm);
\draw [gray,thin] (0,0) circle (5.2cm);
\draw [gray,thin] (0,0) circle (2.8cm);
\draw [gray,thin] (0,0) circle (.42cm);
\node (0,0) {\emoji{gem}};
\foreach \angle / \label / \rotate in
{%0/3, 30/2,
  33/PAR→/-55,
  % 60/1, 90/12, 120/11,
  147/DPL→/55,
  % 150/10, 180/9, 210/8, 240/7,
  270/←CLA/0
  % 300/5, 330/4
}
{
\draw[line width=1pt] (\angle:8cm) -- (\angle:8.2cm);
\draw (\angle:8.6cm) node[rotate=\rotate]{\textsf{\label}};
}
\foreach \angleA / \labelA in
{%0/3, 30/2,
  23/{Dérive Comix},
  13/{Share Back},
  % 60/1, 90/12, 120/11,
  115/{Meaning Map},
  124/{Components},
  127/{Language},
  130/{Pattern},
  140/{Going Meta},
  % 150/10, 180/9, 210/8, 240/7,
  265/{Reinfuse},
  262/{Expertise},
  242/{Roles},
  245/{Functional},
  224/{Control},
  227/{Participant},
  230/{Increase}
  % 300/5, 330/4
}
{
\settowidth{\somelengthA}{\labelA}
\draw (\angleA:6.6cm) node[rotate=\angleA,transform canvas={xshift=\somelengthA}]{\textsf{\labelA}};
}
% PART Bi
\foreach \angleB / \labelB / \corr in
{%0/3, 30/2,
  -0/{Context}/0,
  -4/{Setting}/-3,
%  43/{Contested space},
  % 60/1, 90/12, 120/11,
  109/{Facilitator}/-2,
  104/{Roles}/-7
}
{
\settowidth{\somelengthB}{\labelB}
\draw (\angleB:4cm) node[rotate=\angleB-\corr,transform canvas={xshift=\somelengthB}]{\textsf{\labelB}};
}
% PART Bii
\foreach \angleB / \labelB / \corr in
{%0/3, 30/2,
  -13/{Time Traveler}/0,
  -23/{Reflector}/0,
%  43/{Contested space},
  % 60/1, 90/12, 120/11,
  85/{Linker}/0,
  95/{Analyst}/0,
%  177/{Funding of},
%  182/{Public Space},
  % 150/10, 180/9, 210/8, 240/7,
  220/{Wrinkler}/0,
  207/{Stepper}/0
%  250/{Rebalance},
%  245/{Social Services}
  % 300/5, 330/4
}
{
\settowidth{\somelengthB}{\labelB}
\draw (\angleB:4cm) node[rotate=\angleB-\corr,transform canvas={xshift=\somelengthB}]{\bfseries\scshape{\labelB}};
}
% PART C
\foreach \angleC / \labelC / \corr in
{%0/3, 30/2,
  -33/{Do Your}/0,
  -45/{Research}/10,
  % 60/1, 90/12, 120/11,
  190/{Conversations}/0,
  200/{Structure}/-10,
  % 150/10, 180/9, 210/8, 240/7,
  65/{Begins Now}/0,
  80/{The Future}/-14
  % 300/5, 330/4
}
{
\settowidth{\somelengthC}{\labelC}
\draw (\angleC:1.65cm) node[rotate=\angleC+\corr,transform canvas={xshift=\somelengthC}]{\textsf{\labelC}};
}
% \foreach \angle in {0,90,180,270}
% \draw[line width=2pt] (\angle:1.6cm) -- (\angle:2cm);
% \node[draw=none,font=\tiny,text=red] at (0,.9cm) {TICK-TOCK};
% \draw[rotate=90,line width=2pt] (0,0) -- (-#1*30-#2*30/60:0.7cm); % hours
% \draw[rotate=90,line width=1.5pt] (0,0) -- (-#2*6:1cm); % minutes
% \draw[rotate=90,thin,red] (0,0) -- (-#3*6:1.2cm); % seconds
% \path [fill=red] (0,0) circle (2pt);
%
\end{tikzpicture}%
}

%% Sometimes needed, sometimes not:
% \newcommand\acmPrice[1]{}
%%%

\newfontfamily{\chess}[Color=red]{chess_merida_unicode.ttf}

\usepackage{slantsc}

\setcopyright{rightsretained}
\copyrightyear{2025}
\acmYear{2025}
\acmDOI{10.XXXX/XXXXXXX.XXXXXXX}

%% These commands are for a PROCEEDINGS abstract or paper.
\acmConference[PLoP'25]{PLoP'25: Pattern Languages of Programs 2025}{October 12--15, 2025}{HILLSIDE}
\acmBooktitle{PLoP'25, OCTOBER 12--15, HILLSIDE. Copyright 2025 is held by the author(s)}
\acmPrice{15.00}
\acmISBN{978-1-4503-XXXX-X/18/06}

%\usepackage[Latin,Greek,Emoticons]{ucharclasses}
\usepackage{graphicx}
\usepackage{grffile}
\usepackage{longtable}
\usepackage{wrapfig}
\usepackage{rotating}
\usepackage[normalem]{ulem}
%\usepackage{amsmath}
\usepackage{textcomp}
%\usepackage{amssymb}
\usepackage{capt-of}
\usepackage{hyperref}
\usepackage{fontspec}

\usepackage{emoji}
\setemojifont{Noto Color Emoji}
%\usepackage{coloremoji}

\usepackage{mdframed}
\usepackage{afterpage}

\usepackage{booktabs}

\usepackage[inline]{enumitem}

\usepackage[pagewise]{lineno}
\renewcommand\thelinenumber{\color{red}\arabic{linenumber}}
%\usepackage{xunicode}
\usepackage{natbib}
\usepackage{float}
%\usepackage{xypic}
\usepackage{tikz}
\newcommand{\sensory}{(s)}
\newcommand{\cognitive}{(c)}
\newcommand{\motor}{(m)}
\newcommand*\circlednum[1]{\resizebox{1em}{!}{\tikz[baseline=(char.base)]{\node[shape=circle,draw,inner sep=2pt] (char) {#1};}}}
%\usepackage{amsmath, amssymb}
\def\t{\scriptstyle\triangle}
\def\T{\textstyle\blacktriangle}
\usepackage{placeins}
%\usepackage{starfont}
% \newfontfamily{\alch}{Alchemy}
%\newcommand{\Asclepius}{{\emoji\symbol{"2695}}}
%\newcommand{\Caduceus}{{\emoji\symbol{"2624}}}
\setmainfont{Libertinus Sans}
\newfontfamily{\ser}{Libertinus Serif Italic}
\newcommand{\candidate}[1]{{\ser \scitshape #1}}
\usepackage{bold-extra}
\DeclareRobustCommand{\scitshape}{\fontshape{\scitdefault}\selectfont}
\newenvironment{echo}{}{}
\usepackage{enotez}
\renewcommand{\endnote}[1]{}
\newcommand{\ignorelink}[2][]{#2}
\newcommand{\markbf}[1]{\textsuperscript{\textbf{#1}}}
\setenotez{counter-format = alph, mark-cs = \markbf}
\DeclareInstance{enotez-list}{sverre}{paragraph}{heading={},notes-sep=\baselineskip,format=\normalsize\normalfont\raggedright\leftskip1.8em,number=\makebox[0pt][r]{#1.\ }\ignorespaces,}
\usepackage{epigraph}
\date{}
\title{Patterns of Patterns III}
\hypersetup{
 pdfauthor={Joseph Corneli et al.},
 pdftitle={Patterns of Patterns III},
 pdfkeywords={},
 pdfsubject={},
 pdfcreator={Emacs 30.0.50 (Org mode 9.6.1)},
 pdflang={English}}

\citestyle{acmauthoryear}
\let\cleardoublepage=\clearpage
\begin{document}

\title{Patterns of Patterns III}

\author{Joseph Corneli}
\authornote{Corresponding author, jcorneli@brookes.ac.uk.}
\email{jcorneli@brookes.ac.uk}
\orcid{1234-5678-9012}
\affiliation{%
  \institution{Oxford Brookes University}
  \streetaddress{Gipsy Lane}
  \city{Oxford}
  \country{UK}
  \postcode{OX3 0BP}
}
\affiliation{%
  \institution{Hyperreal Enterprises Ltd}
  \streetaddress{81 St Clement’s St}
  \city{Oxford}
  \country{UK}
  \postcode{OX4 1AW}}

\author{Charles J. Danoff}
\affiliation{%
  \institution{Mr Danoff’s Teaching Laboratory}
 \streetaddress{PO Box 802738}
 \city{Chicago}
 \state{IL}
  \country{USA}
  \postcode{60680}}
\email{contact@mr.danoff.org}

\author{Raymond S. Puzio}
%\authornotemark[1]
\email{rsp@hyperreal.enterprises}
\affiliation{%
  \institution{Hyperreal Enterprises Ltd}
  \streetaddress{81 St Clement’s St}
  \city{Oxford}
  \country{UK}
  \postcode{OX4 1AW}}


%% \author{Vitor Bruno}
%% \affiliation{%
%%   \institution{Milestone English}
%%   \streetaddress{Rua Trieste 170, ap2}
%%   \city{Palhoca}
%%   \state{SC}
%%   \country{Brazil}
%%   \postcode{88132-227}}
%% \email{chief@milestoneenglishcourse.com}

%% % \author{Paola Ricaurte}
%% % \affiliation{%
%% %   \institution{Tecnologico de Monterrey}
%% %   \streetaddress{Calle del Puente 222 Col. Ejidos de Huipulco, Tlalpan}
%% %   \city{Mexico City}
%% %   \country{Mexico}
%% % }
%% % \email{pricaurt@tec.mx}

\author{Sridevi Ayloo}
\email{sayloo@citytech.cuny.edu}
\author{Serge Belich}
\email{serge.belich85@citytech.cuny.edu}
\affiliation{%
  \institution{New York City College of Technology}
  \streetaddress{300 Jay St}
  \city{Brooklyn}
  \state{NY}
  \country{USA}
  \postcode{11201}}

\author{Mary Tedeschi}
\affiliation{%
  \institution{Pace University}
 \streetaddress{One Pace Plaza}
 \city{New York}
 \state{NY}
  \country{USA}
  \postcode{10038}}
\email{mtedeschi@pace.edu}



%%
%% By default, the full list of authors will be used in the page
%% headers. Often, this list is too long, and will overlap
%% other information printed in the page headers. This command allows
%% the author to define a more concise list
%% of authors' names for this purpose.
\renewcommand{\shortauthors}{Corneli et al.}


% (1) Review the intention: what do we expect to learn or make together?

% - Joe: Wanted to walk through the PoP paper with Rebecca, in order to help solidify my own grasp of the concepts and get her feedback.
% - Rebecca: This topic sounded interesting when you mentioned it and I wanted to learn about Pop

% (2) Establish what is happening: what and how are we learning?

% - Indeed we did speed through the paper, Rebecca pointed out a few places where there was friction with the wording or concepts, like ``PEER-TO-PEER'' and also suggested Operational Research and Strategy as an appropriate topic; mentioned “improvisatory” style
% - Interruptions were welcomed!
% - Rebecca: was hesitant to interrupt the narrative
% - This is a bit different from usual IEAI style...

% (3) What are some different perspectives on what’s happening?

% - Joe: appreciate the time Rebecca has put into this a lot, and I also think this was a good way to present the paper
% - Rebecca: I think you assume knowledge in the presentation, and I think you need to assume the listener (if they aren’t in the area) that they don’t know anything.  It wouldn’t be patronising to explain the basic concepts.

% (4) What did we learn or change?

% - Talking to Karl, he would reconise one of the areas (probably) but not necessarily the other two.  Everyone is going to be new to some of these concepts.
% - This was great as a ``final edit'' — we will also be able to edit this paper until December
% - RR: is it your aim to automate?

% (5) What else should we change going forward?

% - Joe: if it would be helpful to RR, I’d certainly be happy to meet again about these ideas
% - Would this (patterns of patterns) to actually be useful for ethical AI?
% - E.g., rethink in the context of moral machines

%%
%% The abstract is a short summary of the work to be presented in the
%% article.
% distributed peer-to-peer networks
\begin{abstract}
Building on earlier installments, this paper re-examines the PLACARD
pattern.  We report on a series of workshops where PLACARD was used to
scaffold collaborative reflection, speculative inquiry, and stimulate
design pattern generation. These accounts are enriched by a comparison
case: virtual workshops carried out with simple AI-based chatbots.  We
discuss limitations and lessons learned from both the human and
multi-agent settings.  We conclude by outlining a future development
strategy at the intersection of AI agents, design patterns, and
institutional governance.

%% COMMENT: (Charlotte) "How used in the real world?
%% COMMENT: (Joe) "Excellent question.  We could just say "not used outside of our workshops" — but actually, we use variations of the patterns that we describe in this paper in our collaborations (in the peeragogy project and beyond).  It would be great to have a short summary of such uses and indeed, that could go into the abstract; but, we'll have to work on it!"


\end{abstract}

%%
%% The code below is generated by the tool at http://dl.acm.org/ccs.cfm.
%% Please copy and paste the code instead of the example below.
%%
\begin{CCSXML}
<ccs2012>
<concept>
<concept_id>10003456</concept_id>
<concept_desc>Social and professional topics</concept_desc>
<concept_significance>500</concept_significance>
</concept>
<concept>
<concept_id>10011007.10011074.10011075</concept_id>
<concept_desc>Software and its engineering~Designing software</concept_desc>
<concept_significance>300</concept_significance>
</concept>
<concept>
<concept_id>10011007.10011074.10011134.10003559</concept_id>
<concept_desc>Software and its engineering~Open source model</concept_desc>
<concept_significance>300</concept_significance>
</concept>
<concept>
<concept_id>10010405.10010481</concept_id>
<concept_desc>Applied computing~Operations research</concept_desc>
<concept_significance>300</concept_significance>
</concept>
<concept>
<concept_id>10010147.10010341</concept_id>
<concept_desc>Computing methodologies~Modeling and simulation</concept_desc>
<concept_significance>100</concept_significance>
</concept>
</ccs2012>
\end{CCSXML}

\ccsdesc[500]{Social and professional topics}
\ccsdesc[300]{Software and its engineering~Designing software}
\ccsdesc[300]{Software and its engineering~Open source model}
\ccsdesc[300]{Applied computing~Operations research}
\ccsdesc[100]{Computing methodologies~Modeling and simulation}


%%
%% Keywords. The author(s) should pick words that accurately describe
%% the work being presented. Separate the keywords with commas.
\keywords{Design Patterns, Pattern Languages, Action Reviews, Futures
Studies, Causal Layered Analysis, Free Software, Peeragogy,
Artificial Intelligence, Anticipation, Doughnut Economics}

%\authorsaddresses{This command processes the author and affiliation and title This command processes the author and affiliation and title This command processes the author and affiliation and title This command processes the author and affiliation and title}

%%
%% This command processes the author and affiliation and title
%% information and builds the first part of the formatted document.
\maketitle


%\clearpage

% \tableofcontents
\section{Introduction}\label{sec:intro}

In an era of increasing global interconnectivity, effective collaboration across cultural, geographical, technical, and disciplinary boundaries is essential. Traditional methods often falter when addressing the intricate layers of context inherent in distributed collaboration.  To address these concerns, we have been exploring a re-theorisation of design patterns using the tools of social intelligence.  First introduced in our paper “Patterns of Patterns: A Methodological Reflection on the Future of Design Pattern Methods” (PoP), the PLACARD pattern was tested in five U.S. and U.K.~workshops as detailed in the sequel “Patterns of Patterns II”~(PoPII) \cite{10.5555/3712039.3712044,10.5555/3721041.3721050}.  To support that work, PLACARD was elaborated into a design pattern language.  It has since been adapted to additional localized settings, engaging diverse participants in hybrid and in-person formats. These sessions leveraged PLACARD to bridge perspectives and envision future modes
of collaboration in rapid prototyping sessions.  We have also been interested in how PLACARD can be used at a slower pace, and in distributed contexts—for example, to scaffold learning in a long-running seminar series—or even more widely, across different spheres of engagement that we have described in this series of papers.  In this paper, we establish another use case, by using PLACARD patterns with virtual artificial intelligence (AI) agents.  We show that design pattern thinking is well-suited to use in developing epistemic AI, by which we understand artificial intelligence systems that model and manage uncertainty, provide explanations, engage in meta-cognition, and facilitate human-AI collaboration. By working between human and computer contexts, we are able to improve our understanding of the underlying patterns of knowledge acquisition, representation, and utilization.  Generative AI tools contributed in other ways to this paper, by
\href{https://www.google.com/url?q=https://groups.google.com/g/peeragogy/c/XM7RKMDdcT4\&sa=D\&source=editors\&ust=1749305683239949\&usg=AOvVaw1GJKEDppYQy3WrqXEi9SV9}{helping us make the first draft} based on their analysis of the two initial papers in this series and our three summary reports. We then took the baton and revised the writing over many subsequent versions to arrive at the cohesive work you are now reading.

\section{Patterns in Use}\label{sec:II}

\subsection{Background}

In its basic design pattern format, PLACARD combines three
complementary approaches for collective learning and design.

\begin{description}
\item[PAR (Project Action Review)] -- A reflective process assessing
  intentions, actions, and outcomes to learn from experience.
\item[CLA (Causal Layered Analysis)] -- A futures studies technique
  analyzing issues across multiple layers---the familiar litany of
  well-known problems, the systems that underpin them, along with the
  prevailing worldviews, and deeper myths that constitute the root
  causes of these problems, along with the possibility of developing
  alternative perspectives.
\item[DPL (Design Pattern Language)] -- A framework for capturing and
  sharing reusable solutions in a structured
  Context-Problem-Solution~format (or similar), further structured by
  links between such elements.
\end{description}

By combining these approaches, PLACARD enables teams to cycle
through reflection, deep analysis, and pattern
creation.  However, this is not necessarily a serial process or
a fixed cycle.  For instance, the generation, use, and analysis
patterns may be ongoing.  In our work a PAR is typically
conducted after a period of work, to reflect on progress made,
however it also serves as an intention-setting exercises might
happen before the next phase begins.  The nominal function of a
CLA is to analyze a situation, but just as often they are used
to describe how that situation can change.

The three papers in the series to date can be brought into schematic
alignment with these methods.

\medskip
\begin{tabular}{lp{.7\textwidth}}
(PoP $\approx$ CLA) &We used CLA to think about the pattern literature as a whole and to retheorise its workings.\\
(PoPII $\approx$ DPL)& We elaborated PLACARD into a pattern language.\\
(PoPIII $\approx$ PAR)& We take a deep look at how these patterns have been used in practice, and take stock of potential next steps in the overall trajectory of research.\\
\end{tabular}

\medskip

Thinking about patterns in pattern-theoretic terms, we uncovered
PLACARD’s modularity.  Methods that achieve the same function as one
of its components can be swapped in and out, without undermining the
integrity of the whole.  For example, PAR could be replaced with the
US Army's After Action Review \cite{afteraction}---which inspired its
basic format in the first place---or by the Feedback Wheel devised by Janet Hurley
\cite{real2008new}. CLA could be replaced by Triple-Loop Learning
\cite{tosey2012origins} or Ecological Systems Theory
\cite{bronfenbrenner-est}.  DPL could be replaced by Elinor Ostrom's
institutions \cite{Ostrom_1990}, or by a catalogue of literary tropes,
and so on.  Each such selection comes with its own ideological
orientation, and typical domain of application.

\subsection{Implementation}

{We implemented PLACARD in multiple settings, tailoring our
  approach to each context:}
\medskip

\begin{description}
\item[AsianPLoP 2024 (Fujisawa, Japan)]{Our “PLACARD Workshop”
  \cite{Danoff2024} was a hybrid focus group with one onsite and two
  remote facilitators (via Jitsi). Participants conducted
  meta-analysis on the conference itself using CLA and DPL to uncover
  underlying challenges, identify linguistic barriers and explore
  growth opportunities. One challenge discussed was how to be more
  inclusive of countries across the Asian continent. The conference
  organizers took a positive step by holding the 2025 event in Taiwan,
  but we cannot state that location decision was influenced by our
  Focus Group. In 2024, onsite attendees helped overcome barriers by clarifying
  for confused participants in Japanese. This was the only event in
  this paper conducted in multiple languages.}{~A growth opportunity
  discussed that learning and networking does not need to stop when
  the conference itself ends. It can continue afterwards as attendees
  cross-pollinate pattern design collectively. The PAR was then
  applied to close the Focus Group and capture the results. During the
  evening conference recap ceremony there was one summary delivered in
  English by a paper author and another in Japanese by an attendee.}
\item[EuroPLoP 2024 (Irsee, Germany)]{A hybrid session titled “AI
  Future Envisioning with PLACARD” \cite{Tedeschi2024} with one
  onsite and three remote facilitators. Participants employed the
  Transfeminist Tech Oracle card game as a creative prompt, then used
  CLA and PAR in tandem to explore speculative AI futures. Due to time
  constraints and abstraction load, some participants expressed
  confusion about the method structure, prompting later pattern
  refinement. A post-session PAR survey was used for reflection after
  the event.}
\item[PLoP 2024 (Washington, USA)]{An in-person “Imagination Run
  Wild” workshop, with three onsite facilitators. Participants were
  invited to “Envision the Future of Collaboration” \cite{ayloo2024},
  using the same card game and CLA.  Another post-session PAR survey
  was used for reflection after the event.}
\item[Experts Among Us 2025 (Keene, New Hampshire, USA)]{A
  stand-alone meeting in-person that gave an opportunity to play the
  same card game with one facilitator. In advance we practiced the
  card game and the facilitator then preselected the cards for
  participants. Attendees were not part of the patterns
  community. ~This session actually facilitated a group discussion
  and demonstrated collaboration. We learned that we had to explain
  the rules and definitions prior to playing the game.}
\end{description}
% We plan to continue this

A common challenge for all of these workshops is rapid skill
acquisition.  We plan to continue this workshop series, for instance
with a EuroPLoP 2025 workshop session.  As mentioned above, this is
a good time to take stock of what we’ve learned so far.

\subsection{Reflexive Analysis Using the PLACARD Pattern Catalogue}

To analyze the workshops beyond surface-level outcomes, we applied a
reflexive methodology grounded in the PLACARD pattern catalogue,
presented in PoPII.  This catalogue is structured into three
interlinked categories:

\medskip
\hspace{-.5cm}\begin{tabular}{ccc}
\textbf{A.}~Identifying Themes (“sensory”)&
\textbf{B.}~Organizing Structure (“cognitive”)&
\textbf{C.}~Making It Actionable (“motor”)\\
\end{tabular}

\vspace{.2cm}
For each workshop,~in this paper we use the
\textbf{A}/\textbf{B}/\textbf{C} structure to retrospectively
assess:

\begin{itemize}
\tightlist
\item Which patterns were implicitly or explicitly activated,
\item Where breakdowns in understanding or facilitation could be explained by missing or misapplied patterns, and
\item Where new patterns could be introduced to resolve to unanticipated tensions.
\end{itemize}

A brief summary is given in Section \ref{sec:III} of this paper.  This
reflexive approach helped us make sense of divergent workshop
dynamics, and the possibility for further patterns that could help
make the workshops more rewarding in general. ~For example, Name the
Frame~was identified in hindsight as something that could have
improved onboarding. Likewise, {\scshape {Harvest Before You
    Close}}~was seen as a missing step in several sessions, and this
pattern would address missed opportunities for extemporaneous pattern
generation.

Rather than enforcing a rigid rubric, the catalogue functioned as a
scaffold for learning from variation. It allowed us to compare
facilitation dynamics, participant responses, and methodological
adaptations across sites---and, crucially, to refine the catalogue
itself through lived experience.

\subsection{A Note on Notation}
As above, we will use all-caps (like PLACARD) for patterns introduced in other sources, and small-caps (like {\scshape{Name the Frame}}) for patterns introduced in this paper, details of which are in the Appendix.

\subsection{Empirical and Design-Based Development of New Patterns}

Across the 2024 and 2025 events, new patterns did not emerge fully
formed.  Indeed, only one new pattern surfaced and named explicitly
during the workshops:

\begin{itemize}
\tightlist
\item[]
  \scshape{Have Multiple Leaders Onsite}{~-- prompted by the hybrid setup at
  EuroPLoP, where limited physical facilitation led to friction. The
  pattern proposes distributed facilitation roles to support hybrid
  presence, reduce coordination load, and maintain session energy.}
\end{itemize}

Other new proto patterns crystallized in our retrospective analysis.
For example:

\begin{itemize}
\tightlist
\item[]
  {\scshape{Name the Frame}} -- (a further specification of
  CONTEXT SETTING from PoPII) this pattern emerged from repeated
  facilitator observations that participants struggled to orient
  themselves without a clear method overview.
\item[]
  {\scshape{Borrow the Situation}} -- inferred from participant improvisation in choosing and reinterpreting prompts when predefined situations felt mismatched or abstract.
\item[]
  {\scshape{Start with Sense}} -- (generalizing {D\'ERIVE COMIX} from PoPII)
    builds on the observation that embodied play or metaphorical
    framing was often the key to unlocking participation
\end{itemize}

{In sum, while few patterns emerged as polished artifacts within the
workshop frame, the pattern language itself evolved through encounter.
The workshops provided empirical grounding for evaluating utility,
exposing limits, and highlighting latent needs. This iterative,
design-based process parallels how pattern languages grow in living
communities---not through top-down invention, but through grounded,
reflective refinement.}

\subsection{Developing a comparison case: Working with patterns computationally}

We noticed that it is possible to ask ChatGPT to run simple
multi-agent ``virtual'' workshops.  These can be compared with the
in-person workshops.  Below are two lightly edited example prompts
that were used for setting up these simulations.

\medskip

\fbox{
\begin{minipage}{.7\pagewidth}
  \textbf{Prompt 1 Given to Generative AI Program}

{}

{I want you to create a multi-agent system with 15 different
  ChatGPT-based agents. ~I want you to assign each of them the name of
  a distinct animal, and each of the agents will have some of the
  typical characteristics of the animal it is named for. ~Then I want
  you to walk through the 30 “Peeragogical Exercises for a Seminar on
  Epistemic AI” [elided] in a virtual seminar. ~The
  final exam is a dialogue between these animals, about the following
  prompts. ~Please have as much discussion as you’d
  like among the agents before you give me their final team
  presentation answering the questions.}

{}

{What is epistemic AI?}

{}

{How does mathematics relate to this vision?}

{}

{What does `peeragogy' mean, and how does it relate to this vision?}

{}

{What kind of multi-agent has created this answer?}

{}

{Why is this virtual seminar of historical importance?}
\end{minipage}}

\fbox{
\begin{minipage}{.7\pagewidth}
  \textbf{Prompt 2 Given to Generative AI Program}

  Remember how we did some simple multi-agent workshops before?  Let's
  simulate the architecture described the paper with some
  ChatGPT-based agents.  This time, rather than basing them on
  animals, we are going to base the agents on a diverse group of
  Research Fellows in AI.  Let's only consider 10 such agents.  Please
  include the “Ant” role from the workshop with animals, she will
  record what happens in which workshop session.  Please also include
  the “Dog” role from the animal workshop, who will act as a primary
  reporter.  The Fellows will be set the task of designing a series of
  workshops using the PLACARD method and associated patterns, and the
  outcome of the series should be a detailed plan of work for a
  research project on epistemic AI.  You should do deep research to
  make sure that the discussions are plausible.
\end{minipage}}

\medskip

\section{Patterns as Technical, Epistemic, and Reflective Infrastructure}\label{sec:III}

\subsection{Reframing Patterns as Sense-Making Scaffolds}

Patterns are often described in instrumental terms: reusable solutions
to recurring problems in context.  As we look into how they work, we
see there’s more to it.  One aspect of this is the use-versus-mention
distinction: clearly, a physical alcove is not the same as Christopher
Alexander's “ALCOVES” pattern \cite{alexander1977a}.  And at both
levels, there’s a degree of explicitness.  When does an indentation
become an alcove?  When does a pattern become part of a language?

The {\scshape{Start with Sense}} pattern characterizes various ways of
inviting participants to begin inquiry with embodied, perceptual, or
intuitive grounding. This creates conditions under which problems and
possibilities become intelligible.

Across the A/B/C catalogue, many patterns introduced earlier and from
the Appendix operate similarly:

\begin{itemize}
\tightlist
\item[]
  {\scshape{Name the Frame}}~helps surface the underlying structure of an activity or method.
\item[]
  {\scshape{Borrow the Situation}}~allows participants to reinterpret or co-create the context of inquiry.
\item[]
  {\scshape{Meta-Reflection Loop}}~supports collective understanding of what just happened and why.
\end{itemize}

These are cognitive footholds, ways of managing ambiguity and
distributing attention. They help groups compose a shared reality,
especially when dealing with complex or unfamiliar material.

From this perspective, the value of a pattern lies not just in its
repeatability but in its sensitivity to situated meaning-making. A
good pattern doesn't just say ``do this''; it says ``look here, attend
to this tension, notice this affordance.'' It invites participants to
see the world in terms of actionable differences.

Reframing patterns this way brings PLACARD into closer dialogue with
traditions such as:

\begin{itemize}
\tightlist
\item
  Phenomenology—emphasis on situated perception \cite{phenomenology-intro}
\item
  Participatory design—emphasis on co-articulation of problems \cite{participatory-design}
\item
 Grounded theory—emphasis on emergent categorization \cite{glasser}
\end{itemize}

It also prepares the ground for computational application. For
artificial agents, much as for humans, sense-making begins with
activation of relevance---the ability to determine what matters in a
given context. In PLACARD, patterns serve as activation scaffolds:
semi-formal cues that guide what should be noticed, named, or
negotiated.  Each of its layers---A/B/C---corresponds to a distinct
mode of learning.  Crucially, design patterns---or, indeed, any
substitute schema that plays the same role---achieve the function of
linking these modes, enabling transitions between experience, inquiry,
and abstraction.

We propose that patterns support at least three interdependent forms
of learning:

\begin{enumerate}
\tightlist
\item
  Reflective Learning \cite{Boyd1983}: Patterns such as
  PROJECT ACTION REVIEW {~and }{\scshape{Meta-Reflection
      Loop}} support meaning-making after action. They help
  participants surface experiences, identify what shifted, and
  consider next steps. This mode is retrospective and grounded in
  collective memory and sense-making.
\item
 Design Learning \cite{BurdickDesign}: Patterns like {\scshape{Borrow the
     Situation}}~or {\scshape{Pattern the Play}}~create
 scaffolds for reimagining problems, proposing alternatives, and
 testing ideas.  Design learning is generative, situated in
 ambiguity. Patterns here serve as framing devices that guide
 creativity while constraining chaos.
\item Model Learning: Patterns like {\scshape{Name the Frame}} and
  {\scshape{Harvest Before You Close}} deploy abstraction---capturing
  structural insights from activity. This form of learning supports
  knowledge representation and the translation of local insights into
  portable conceptual tools (e.g., \cite{ariyaratne1977a}).
\end{enumerate}

Patterns enable movement between these modes. For example:

\begin{itemize}
\tightlist
\item
  {A reflection may generate a design insight that prompts reframing.}
\item
  {A design intervention may require formal modeling to share or
  extend.}
\item
  {A model may provoke a return to embodied or emotional grounding.}
\end{itemize}


\subsection{Summary of the evidence base from our in-person workshops} \label{sec:workshop-evidence}

Table \ref{workshop-summary} summarizes the use of PoPII patterns in
our workshops.  Further details are the Appendix.

\begin{longtable}[]{@{}p{.3\textwidth}p{.5\textwidth}@{}}
\toprule\noalign{}
\endhead
\bottomrule\noalign{}
\endlastfoot
{Category}/{Patterns Used Well} & {Needs Refinement} \\[2mm]
\textbf{A.}/DÉRIVE COMIX, DO YOUR RESEARCH, CONTEXT SETTING, PAR & \emoji{warning} \textbf{Overrun} — PAR provides shared cohesion, when used; our PAR-inspired survey has a different feel.  Context-setting could also be more cohesive: {\scshape Name the Frame}. Explore more creative, embodied PAR. \\[.2mm]
%%
\textbf{B.}/{GOING META, FACILITATOR and FUNCTIONAL ROLES} & \emoji{warning} \textbf{Mixed} — time + onboarding were limits. Introduce meta-language tools earlier and more clearly. Test whether embodied cues help people get to worldview/myth. \\[.2mm]
%%
\textbf{C.}/{REINFUSE EXPERTISE} & \emoji{warning} \textbf{Implicit} — patterns used, not made.  Stronger follow-through on actionable steps;
{\scshape Harvest Before You Close}. Explicitly support pattern emergence from cards/music activities. \\
\caption{Summary of patterns used in workshops and potential for improvement\label{workshop-summary}}
\end{longtable}

\section{Design Patterns for Pattern-Competent Agents}\label{sec:IV}

The PLACARD methodology, as developed and refined across multiple
participatory workshops, demonstrates the power of design patterns not
only as tools for structuring action, but also as scaffolds for shared
sense-making and reflection. Building on this foundation, we now
consider a forward-looking design space: the development of
pattern-competent artificial agents capable of engaging with design
patterns as part of collaborative inquiry.  These agents would be
expected to:

\begin{itemize}
\tightlist
\item
  {Recognize situations where a pattern might apply,}
\item
  {Use flexiformal representations of pattern logic,}
\item
  {Operate as a collective, i.e., to form multi-agent systems (MAS)}
\item
  {Participate in reflection loops that evolve the infrastructure
  itself.}
\end{itemize}

\subsection{Recognizing Pattern-Relevant Situations}

{The capacity to recognize situations where a pattern might apply lies
at the heart of pattern competence. This is not a trivial matching task.
It requires the ability to perceive and interpret contextual cues, to
identify when a recurring structure or problem-type is emerging, and to
discern salience: which aspects of a situation matter, and why. In
participatory settings, such recognition is informed by experience,
tacit knowledge, and social interaction. When developing artificial
agents, the challenge is to develop mechanisms---symbolic, statistical,
or hybrid---that allow }{for similarly situated recognition.}{~Here's an
example showing how this might work for today's computer systems:}

\begin{quote}
\emph{Example:}~{A chatbot trained to handle customer service inquiries
recognizes that a user is ``frustrated,'' even if the words don't match
prior examples exactly---but the tone, pacing, and word choice are
similar to previously tagged ``frustrated'' interactions.}

\emph{Mechanism:}{~A statistical similarity model (e.g., embedding distance +
fine-tuned classifier) matches this new situation to prior ones and
adjusts the agent's response accordingly (e.g., a more empathetic tone or
an escalation offer).}
\end{quote}

{Next-generation systems might deal with the same scenario in more flexible ways, for instance, not only on the basis of stimulus-response pairs, but more fluidly---for instance, understanding how it was that the customer became frustrated and initiating changes to the company policy accordingly. This more meaningful form of `pattern recognition' requires readiness to see and make sense of structure, and to reframe experience in meaningful, actionable terms --- rather than to set data aside as a block, or (which is much the same) to cope with it in a purely reactive way.

\subsection{Flexiformal Representation of Pattern Logic}

{To support agent-based reasoning, patterns must be expressed in a
format that is both structured and open-ended. Drawing on the modified
pattern structure used in PLACARD---\textbf{Context}{\ldots{} }\textbf{If}{~\ldots{} }{(\textbf{But})}{\ldots{}
}\textbf{Then}{\ldots{}}\textbf{Because}{\ldots{}}---each clause can be understood as a layer of scoping and activation:}

\smallskip
\begin{description}
\tightlist
\item[Context:]{~a broad frame (task domain, discourse setting) that
  activates relevant subsystems,}
\item[If condition:]{~a salient event or state within that frame,}
\item[But:]{~a tension or constraint that introduces conflict,}
\item[Then:]{~a candidate action or communicative move.}
\item[Because:]{~}While logic and reasoning have been core topics in
  computer science from its outset \cite{sowa-logic}, further research
  is needed to determine what sorts of reasoning---clearly combining
  abductive, deductive, inductive, and analogical modes---and what
  types of logic---classical, fuzzy, modal, monotonic,
  non-monotonic---are used by humans to think about patterns.
\end{description}

For now, we refer to the overall approach as ``flexiformal''
\cite{6481008}, because this leaves the door open to combining
different formal systems in a suitably flexible fashion. Simply adding
more patterns won't be enough to implement human-like reasoning in
pattern-capable systems.  For one thing, the patterns need to be used
to dynamically allocate cognitive resources.

\subsection{Reflection Loops and the Evolution of Epistemic Infrastructure}

To sustain meaningful pattern use, agents must also be able to
reflect.  This means embedding meta-level structures like the PAR into
the environment itself.  Reflection loops should support the capture
of learning, the revision of pattern catalogues, and the emergence of
new design insights.

In future architectures, agents and humans alike can annotate, revise,
and remix patterns, and participate in the shared cultivation of
‘living’ knowledge systems, in which patterns are part of a
generative infrastructure for collective intelligence.  We’ve
previously imagined patterns being used within a “Discovery Machine”
\cite{Tedeschi2024}.

\subsection{Rethinking the Pattern Catalogue}

The PLACARD A/B/C pattern catalogue was created with human
participants in mind. It emphasizes social affordances, embodied
prompts, and facilitation moves. While some of these translate readily
into multi-agent contexts---e.g., “{\scshape {Meta-Reflection Loop}}”
or “{\scshape {Pattern the Play}}”---others rely on human sensory
input or informal framing strategies that are not directly available
to AI systems.

Thus, we propose a tripartite model of pattern types needed for
pattern-competent agents, generalizing the layers we used previously.

\subsubsection*{1. Interactional Patterns (generalizing the sensory layer, e.g., PAR)}

{PAR structures collaborative reflection and coordination. Interactional
patterns in general guide agents in playing roles, reflecting, and
negotiating meaning in collaborative settings---essentially the
micro-level behaviors that mirror PAR's inquiry loop. }{Examples
include:}

\begin{itemize}
\tightlist
\item[]
  {\scshape{Role Reveal}}{: Adopt explicit roles in multi-agent dialogue.}
\item[]
  {\scshape{Meta-Reflection Loop}}{: Periodically engage in structured
  self-review.}
\item[]
  {\scshape{Timebox the Core}}{: Manage resource allocation and attention
  windows.}
\end{itemize}

These patterns require some translation to make it understandable to
the machine, e.g., encoding roles as functions in MAS architectures or
structuring turn-taking rules.

\subsubsection*{2. Cognitive Patterns (expanding upon the existing cognitive layer,
e.g., CLA)}

CLA unpacks multiple layers of meaning---litany, system, worldview,
myth. These cognitive patterns help agents reason across contexts and
detect tensions, embedding depth and nuance in internal
reasoning. They are interpretive structures, supporting internal
reasoning and pattern use:

\begin{itemize}
\tightlist
\item[]
  {\scshape{Pattern Activation}}{: Given a context and goal, retrieve applicable
  patterns.}
\item[]
  {\scshape{Tension Detection}}{: Use heuristics to identify contradictions or
  bottlenecks.}
\item[]
  {\scshape{Candidate Move Generation}}{: Translate pattern suggestions into
  operational plans.}
\end{itemize}

\subsubsection*{3. Infrastructural Patterns (generalizing the motor layer, e.g., DPL)}

DPL captures reusable structures for action. These infrastructural
patterns define persistent templates and system-wide protocols that
support reuse, feedback, and evolution of the broader system
design. For example these are 3 meta-patterns that could enable
engagement with patterns:

\begin{itemize}
\tightlist
\item[]
  {\scshape{Shared Memory Architecture}}{: Persistent storage of pattern use and
  modification. Putting patterns where people can get them, like the
  catalogue, akin to the SCRAPBOOK from Patterns of Peeragogy \cite{10.5555/3124497.3124531}}
\item[]
  {\scshape{Feedback Rhythms}}{: Protocols for scheduling reflection
    and pattern revision.}
\item[]
  {\scshape {Pattern Diffusion}}{: Mechanisms for propagating new or
    adapted patterns across agent collectives.}
\end{itemize}

They govern the architecture of the agent system much like DPL governs
the architecture of human practice.

\subsection{Desiging Environments for pattern-competent MAS}

In addition to broader contextual information, relevant multi-agent system (MAS) environments would include the social and cognitive dynamics of inquiry: coordination, negotiation, reflection.  Such environments have the capacity to support multiple, interacting Baldwin effects---individual learning loops that influence each other over time as the agents and their interactions evolve \cite{Redko}.  This distributed adaptation mirrors cultural evolution: strategies become shared, revised, and recontextualized.  Rather than simply evaluating \emph{success} in terms of task completion or efficiency, MAS environments invite questions about “how” solutions are created.

This reflection constitutes suggests a key amendment to PLACARD.
Whereas PLACARD was described in classic psychological terms
\cite{Hilgard1980} — which are also taken over by roboticists
\cite{cassimatis2004integrating} — it does not specifically talk about
the environment in which the agents it considers would be situated.
In robot terms, for example, we would expect an undersea robot and a
Mars rover to have very different functionality, even if they have the
same theoretical underpinning.  This way of thinking suggests that the
PLACARD pattern language should be extended to be an “\textbf{A}/\textbf{B}/\textbf{C}/\textbf{D}
catalogue”, where the fourth section contains \emph{environmental} or
\emph{developmental} patterns.  With this in mind, in addition to
further detail on the candidate patterns for pattern-competent agents
— as outlined above — the appendix also includes some patterns that
would guide the design of their working environments.  

\section{Virtual Multi-Agent Workshops in the Spirit of PLACARD}\label{sec:V}

To explore the idea of pattern-competent agents developed Section \ref{sec:IV}, we conducted a series of virtual workshops using fictional agents within ChatGPT. ~Using the PLACARD method, those agents rotated through breakout sessions, engaged in pattern-guided inquiry, and made use of a curated and emergent pattern catalogue.  It is worth emphasizing that the workshops were computer-aided design fictions (we have previously explored this way of working “by hand” \cite{fictional-peeragogical}, and we’re aware that fictional works can have unexpected real-world ramifications, e.g., \cite{enwiki:1290575156}).  ChatGPT can simulate plausible dialogues, behaviors, and reflective loops based on design patterns, but it cannot build and run custom agent architectures.  All `agents' were textual personas that were simulated by one model.  The agents were plausibly able to discuss the design patterns and employ them in light-weight design exercises.  Below is an excerpt from one such workshop,
in which several virtual agents begin collaboratively applying the PLACARD methodology to plan a research project on epistemic AI.  The agents have somewhat stereotyped fictional biographies which to some extent shape their responses (e.g., “Carlos – Epistemology \& Systems Philosophy”; “Mei – Participatory Design \& Feminist STS”).
\smallskip

\fbox{
\begin{minipage}{.7\pagewidth}
\textbf{Excerpt: PLACARD Planning Dialogue on Epistemic AI}
\smallskip

\noindent\textbf{Carlos:} I propose we begin with CONTEXT SETTING. If we're designing a research project on epistemic AI, we need a shared sense of what “epistemic” means in our context.

\noindent\textbf{Priya:} Agreed. For me, it’s about how agents—human or artificial—come to know, not just calculate. So I’d emphasize design patterns that support interpretive, situated reasoning.

\noindent\textbf{Raj:} Right. Maybe we could use {\scshape Borrow the Situation}—look at real-world cases of failure or success in knowledge systems. Misinformation management? Scientific discovery pipelines?

\noindent\textbf{Sofia:} I think we need to {\scshape Name the Frame} explicitly. Are we researching tools for knowledge validation? Or institutional models for human-AI epistemic partnerships?

\noindent\textbf{Mei:} Maybe both. What if we used {\scshape Pattern the Play} to design speculative workshop sessions? Each session could simulate a different epistemic tension—uncertainty, disagreement, novelty—and trial agent responses.

\noindent\textbf{Yuki:} And we need some kind of {\scshape Meta-Reflection Loop}. A structure where, after each workshop, we reflect not just on what we learned, but how the patterns performed. This could feed into our developmental arc.

\noindent\textbf{Carlos:} So the goal of the series isn’t just content generation—it’s to test, refine, and expand the pattern catalogue itself?

\noindent\textbf{Sofia:} Exactly. And that aligns with the DO YOUR RESEARCH pattern: we’re generating grounded material and modeling the tools that structure knowledge formation.

\noindent\textbf{Raj:} We’ll need a pattern-aware toolchain too—maybe a {\scshape Pattern Knowledge Repository} that evolves with each workshop. Otherwise we lose coherence.

\noindent\textbf{Priya:} Let’s add a {\scshape Timebox the Core} phase to every session. This gives us just enough structure to avoid sprawling, without over-engineering.

\noindent\textbf{Mei:} I’ll draft a plan using PLACARD as a scaffolding template. We can each take ownership of one layer—PAR, CLA, DPL—and develop sub-patterns and protocols that fit.
\end{minipage}}

\section{Discussion}\label{sec:discussion}

\subsection{Limitations and Lessons Learned}

While the outcomes of our PLACARD-based workshops (both in-person and
simulated) were promising, several limitations emerged that highlight
areas for refinement---both in the method itself and in our
implementation of it.  Generally, these limitations correspond to
lessons that could be applied in future work.

\begin{longtable}[]{@{}p{.45\textwidth}p{.45\textwidth}@{}}
\toprule\noalign{}
\endhead
\bottomrule\noalign{}
\endlastfoot
{Limitation} & {Lessons Learned} \\ \hline
%%%%%%%
{Cognitive demands of CLA}{: The abstraction and depth of Causal
  Layered Analysis (CLA) often overwhelmed participants, especially
  when scaffolding was limited.}
%%%
& {Provide lightweight onboarding tools---such as mini-patterns,
    worked examples, or scenario walkthroughs---to ease entry into
    layered analysis.} \\ \hline
%%%%%%%
{Short workshop windows (60--90 mins)}{: Limited time forced
  trade-offs, with later phases (e.g., PAR) often rushed or skipped.}
%%%
& {Recognize {\scshape{Timebox the Core}} as a necessary pattern---but
    pair it with mechanisms for asynchronous reflection or follow-up
    synthesis.} \\ \hline
%%%%%%%
{Challenges in hybrid facilitation}{: Remote facilitators struggled to
  engage with group energy and activities; in-person facilitators were
  overburdened.}
%%%
& {Adopt and refine the pattern } {\scshape{Have Multiple Leaders
    Onsite}}{, and explore interactional choreography to balance
  facilitation across modalities.} \\ \hline
%%%%%%%
{Sequential framing constrained emergent inquiry}{: Participants
  sometimes forced their intuitive insights into retroactive
  justifications.}
%%%
& {Allow more modular, remixable process designs. Embed
    ``{\scshape{Improvise to Test}}'' and ``{\scshape{Pattern the
        Play}}'' to legitimize non-linear sense-making.} \\ \hline
%%%%%%%
{Deliberative bias limits affective and tacit knowledge}{: Structured
  methods sometimes crowded out emotional or disruptive forms of
  insight.}
%%%
& {Integrate methods or prompts that scaffold expressive, intuitive,
    and affective contributions---e.g., through metaphor, narrative,
    or art-based reflection.} \\ \hline
%%%%%%%
{Risk of premature formalism}{: Design patterns encouraged
  generalization at the expense of context specificity.}
%%%
& {Emphasize pattern annotations and situated stories. Use pattern
    revision cycles to adapt and diversify rather than standardize.}
\\ \hline
%%%%%%%
{Logistical overhead of documentation}{: Without dedicated roles or
  tooling, PLACARD's emphasis on shared surfaces was burdensome.}
%%%
& {Support tool development and role formalization (e.g., for pattern
    capture). Design light-touch alternatives for low-resource
    contexts.} \\ \hline
%%%%%%%
{Simulated agents were archetypal, not adaptive}{: The multi-agent
  models lacked real-time learning or memory; they played out scripted
  reasoning paths.}
%%%
& {Treat the current set of simulations as }{design probes}{~rather
  than predictive models. Use them to explore design space and pattern
  viability, not behavioral realism.} \\ \hline
%%%%%%%
{Idealized dialogues lacked messiness}{: No true misalignment,
  conflict, or unresolved disagreement appeared in the simulations.}
%%%
& {Future work should inject structured friction---conflicting goals,
    values, or timing---to better stress-test interactional patterns.}
\\ \hline
%%%%%%%
{No persistent identity or memory across sessions}{: Agents did not
  evolve across workshops, limiting insight into long-term learning.}
%%%
& {Incorporate persistent memory structures and episodic traces to
    study cumulative reflection, pattern evolution, and
    institutionalization.} \\ \hline
%%%%%%%
 {Language model constraints
  shaped outcomes}{: Agent behaviors emerged from a coherence-biased
  generative system, not open-ended cognition.}
%%%
& {Simulations must be interpreted with epistemic humility. Their role
    is speculative, not definitive---ideal for surfacing design
    tensions, not modeling reality.} \\ \hline
\end{longtable}

\subsection{Related Work}
Looking forward, the next phase of PLACARD's evolution could be
a~platform for cultivating design-pattern-capable agents, communities,
and learning environments. PLACARD should be understood not as a
finished method but as a generative infrastructure: a living commons
of epistemic tools, shaped by their user communities. ~These insights
can be combined with Elinor Ostrom's Design Principles for Governing a
Commons to serve as a lens for evaluating and extending PLACARD,
beyond the ``agent'' level to the ``environment'' level. On this
point, we can point to some work: \citet{10.1145/2382570.2382575}
explore how institutions can self-organize by formalizing
socio-economic principles that guide agent behavior within Multi-Agent
Systems (MAS).  Drawing inspiration from Elinor Ostrom's work, they
propose an axiomatic design approach~that captures principles such as
proportionality, accountability, and collective choice in a
machine-interpretable form. Where Pitt et al. emphasize axioms as
foundational building blocks, our PLACARD-based approach leverages
design patterns as intermediate representations that bridge human
practice and machine reasoning. ~In a more recent paper,
\citet{10.1162/isal_a_00800} present a methodological framework for
integrating power sensitivity into the design of cyber-physical
systems by drawing on second- and third-order cybernetics. They
propose pattern-based design methods that foreground the observer's
position, across different layes of embeddedness---with beneficial
applications of technology ranging from ``Saves Physical and Cognitive
Resources'' to ``Solves Complex Problems'' to ``Deals with Social
Constructions'' to ``Mitigates Existential Crises''.

\section{Conclusion}\label{sec:conclusion}

Looking back over the limitations and lessons learned, it’s worth
drawing out one top-level remork.  ChatGPT is adept at text
processing, and we have put it to use significantly in our work on
this paper.  It was able to generate and discuss design patterns that
would describe pattern-capable agents.  However, it was not able to
actually use these patterns to build and run pattern-capable agents.
We hope others will be interested in using LLMs to work with design
patterns—but we also hope that such usage will serve as a landmark
rather than a destination.

Including a computational use case within the paper suggested a key
amendment to PLACARD: adding a fourth environment dimension to the
pattern catalogue . Human-facing and Human-Computer Interface-facing
patterns should eventually be extended similarly.

For now, to close this paper, we put forward the following CLA, which
brings together the ``human'' and ``AI'' use cases and outlines a
trajectory of future work.  In the context of this paper, the CLA
layers can aligned with the expanded view of PLACARD that we have
discussed—encompassing \emph{sensory} (litany), \emph{cognitive}
(system), \emph{motor} (worldview), and \emph{environment} (myth)
dimensions.

% ✴️
\subsection*{\emoji{eight-pointed-star} Litany: Known Problems and Existing Solutions}

These are the most visible and widely discussed elements of current AI
discourse: the promise and perils of large language models (LLMs) and
the early success of LLM-associated code-writing agents.

\paragraph{Language Model Substrate}

These foundational models (e.g., GPT-4) provide generative
capabilities, statistical reasoning, and natural language interfaces.
However, they lack structure, grounding, and memory.

\paragraph{Code-Writing \& Execution Agents}

Many simple agents today (e.g., auto-scripting agents, plugin
interfaces) use LLMs to write and execute code or perform isolated
tasks. These are effective but brittle, with limited reasoning
transparency or contextual learning.

% ⚙️
\subsection*{\emoji{gear} Systemic Layer: Epistemic Infrastructure via Patterns}

To move beyond isolated capabilities, we introduce pattern-based
epistemic scaffolding~for multi-agent collaboration. This makes the
system reflexive, composable, and governed by design principles.

\paragraph{Pattern-Aware Dialogue Agents}

Agents trained to recognize and apply design patterns (from the
PLACARD catalogue) in their interactions. They exhibit situated
competence~and can engage in structured reasoning, scenario
exploration, and inquiry.

\paragraph{Pattern-Reflective Meta-Agents}

These agents do not just use patterns---they also monitor, revise, and
propose new ones, participating in iterative design and reflective
cycles. They enable epistemic learning~at the system level,
coordinating multi-agent behavior across contexts.

% 🌍
\subsection*{\emoji{globe-showing-asia-australia} Worldview Layer: Institutional Design through MAS}

Inspired by Elinor Ostrom's principles for governing commons, we treat
the MAS as a social-technological institution~that evolves through
interaction.

\paragraph{Multi-Agent Institution Designers}

This layer operationalizes developmental patterns~that support
reflexivity, feedback, nested structures, and shared norm evolution
across MAS environments. These agents co-design systems of engagement,
monitor norm congruence, and resolve governance disputes---just as
human institutions do. They instantiate a worldview in which
intelligence is not just personal or cognitive, but systemic and
collective.

% 🕊
\subsection*{\emoji{dove} Myth Layer: Human Institutions as Living Models}

Beyond technical systems, our proposal invites reflection on how
real-world institutions---academic, policy, scientific---can
themselves become more adaptive, transparent, and pattern-literate
through interaction with these tools.

\paragraph{Real-World Interface}

This is the last layer because it is grounded in recursive reform.
Humans build tools to simulate and improve human institutions---not to
replace them, but to co-evolve.  This layer anchors the entire
architecture in civic legitimacy, ethical accountability, and
reflective participation. Our claim is not merely that we can control
the system, but that we can use it to help re-pattern the world~we
inhabit.  By working with agents that are both pattern- and
institution-capable, we gain assurances not only about avoiding
``run-away AI'' scenarios, but about avoiding other tragedies of the
commons.

\section{Acknowledgements}\label{sec:ack}
We wish to acknowledge discussions with Cris Viseu around our 2025
EuroPLoP workshop, which continues the line of work described in the
paper.

This paper also benefited from extensive interaction with OpenAI’s
ChatGPT.  Its capacities for text generation, summarisation, pattern
elicitation, and recursive prompting played a formative role in
shaping both the structure and substance of this work.  We also
gratefully acknowledge the unnamed authors of the texts on which
ChatGPT was trained.  Responsibility for the interpretations,
appropriations, and conclusions presented here remains entirely with
the human authors.



\section*{APPENDIX}

This appendix supports the analysis in the paper. It contains:

\begin{enumerate}
\item A pattern-based analysis of the 2024 workshops, using the patterns from PoPII, and exploring the extent to which these patterns were used (implicitly or explicitly).  A summary appears in Table \ref{workshop-summary} in Section \ref{sec:workshop-evidence}.
\item Potential new PLACARD patterns, mentioned in the paper.
\item Candidate Patterns for Pattern-competent agents.
\item Candidate Patterns for Pattern-competent Agent Environments.
\end{enumerate}

\subsection*{Pattern analysis of the 2024 workshops}

One immediate result was the identification of of a proto-pattern,
“Have Multiple Leaders Onsite,” addressing challenges in
hybrid-facilitated events:

\begin{quote}
\textbf{Context:} Interactive conference workshops with both in-person and
remote participants.

\textbf{Problem:} A single onsite facilitator struggles to manage breakout
groups, technology, and overall flow, leading to gaps in support and
reduced engagement (especially for remote attendees).

\textbf{Solution:} Deploy multiple co-facilitators physically onsite to
coordinate activities, handle logistics, and engage with
participants. Onsite leaders can divide roles (e.g., one per breakout
table or task), ensuring better guidance and responsiveness, while
remote facilitators focus on online participants.
\end{quote}

% “how”

\medskip

\subsubsection*{A.~Identifying Themes}

\paragraph*{DÉRIVE COMIX}

\noindent \textbf{Application:} Participants used cards (object, value, body/territory, situation) as interpretive prompts to spark imagination—mirroring  urban dérive in a conceptual sense. \textbf{Observation:} The card game facilitated divergent thinking akin to exploring unfamiliar terrain.

\paragraph*{CONTEXT SETTING}

\noindent \textbf{Application:} Early slides and explanations introduced key concepts (e.g., design patterns, CLA). \textbf{Opportunity:} Participant feedback highlighted a need for clearer upfront orientation. Improved alignment with this pattern could increase accessibility.

\paragraph*{DO YOUR RESEARCH}

\noindent \textbf{Application:} Past workshop experiences were integrated into the design; earlier uses of CLA and PAR informed this session.  \textbf{Example:} Reference to Anticipation 2019 and prior events illustrates “pre-research” informing present design.

\subsubsection*{B.~Organizing Structure}

\paragraph*{GOING META}

\noindent \textbf{Application:} The project itself exemplifies recursive methodology—using PLACARD to refine PLACARD. \textbf{Highlight:} Participants were prompted to reflect on and imagine both the methodology and the subject (AI futures).

\paragraph*{PATTERN LANGUAGE COMPONENTS}

\noindent \textbf{Partial:} The workshop design showcases layering (CLA + DPL + PAR) but participants struggled to parse the conceptual load quickly. \textbf{Recommendation:} Use lighter keywords (HOWEVER, BECAUSE, SPECIFICALLY) to scaffold understanding incrementally.

\paragraph*{FACILITATOR ROLES / FUNCTIONAL ROLES}

\noindent \textbf{Application:} Hybrid facilitation with distinct roles (onsite leader, remote supporters) models distributed facilitation. \textbf{Opportunity:} Could be made more explicit—e.g., naming “Linker,” “Reflector,” or “Wrinkler” roles to participants.

\subsubsection*{C.~Making it Actionable}

\paragraph*{THE FUTURE BEGINS NOW}

\noindent \textbf{Application:} The card game generated imagined futures that participants shared and developed. \textbf{Gap:} While creative ideas emerged, follow-through actions or small-scale pilots were not described. An actionable step could have enhanced continuity.

\paragraph*{PROJECT ACTION REVIEW (PAR)}

\noindent \textbf{Application:} Strong use of the PAR pattern at the end of the session to reflect, share learnings, and gather insights. \textbf{Insight:} Participants gained confidence and intended to reuse techniques—an indicator of success.

\paragraph*{REINFUSE EXPERTISE}

\noindent \textbf{Application:} Participants brought diverse professional and cultural perspectives. Expertise was implicitly reintegrated via the shared reflection on AI futures and governance. \textbf{Suggestion:} A more explicit arc—temporarily setting aside expertise to foster common ground—could enrich this pattern’s application.

\subsection*{Potential new PLACARD patterns}

\subsubsection*{A.~IDENTIFYING THEMES}

\paragraph*{{\scshape Start with Sense}}

\noindent\textbf{Context:} You are facilitating a workshop intended to generate new ideas or shared understanding.
\textbf{If:} You want to activate creativity and presence,  \textbf{But:} participants arrive cold or disconnected from each other and the topic,
\textbf{Then:} Begin with a sensory or embodied activity (e.g., rhythm, image-based storytelling, gesture) that creates shared energy and primes pattern recognition.
\textbf{Evidence base from related literature:} Techniques such as "story circles" (participatory arts), embodied methods in design ethnography, and rhythm-based facilitation in music education all affirm the value of grounding group attention through shared sensory engagement. These strategies mirror the function of PAR alternatives that aim to surface and align affective states as a precondition for inquiry.

\paragraph*{{\scshape Name the Frame}}

\noindent\textbf{Context:} You are introducing participants to a method that is unfamiliar or complex (like PLACARD).
\textbf{If:} You want them to participate actively,  \textbf{But:} the framing is opaque or abstract,
\textbf{Then:} Explicitly describe the method, its purpose, and how today’s session fits into the broader picture.
\textbf{Evidence base from related literature:} This mirrors practices from facilitation toolkits like Liberating Structures (e.g., "Purpose-to-Practice") and the role of explicit framing in adult learning theory, where naming the structure of engagement is key to building learner autonomy and motivation.

\paragraph*{{\scshape Borrow the Situation}}

\noindent \textbf{Context:} You are using prompts (cards, challenges, scenarios) to guide group inquiry.
\textbf{If:} You offer a situation that feels flat or generic,  \textbf{But:} participants want something meaningful to anchor their play,
\textbf{Then:} Invite participants to select, remix, or define their own situation to explore.
\textbf{Evidence base from related literature:} User-centered design emphasizes co-defining the design challenge. Futures methods such as the "Futures Triangle" and participatory foresight often encourage the co-creation of situational prompts to enhance contextual fit and participant investment.

\subsubsection*{B.~Organizing Structure}

\paragraph*{{\scshape Pattern Priming}}

\noindent \textbf{Context:} You are working with participants who may be new to design patterns.
\textbf{If:} You want people to identify and name patterns,  \textbf{But:} the idea of a "pattern" is still abstract,
\textbf{Then:} Seed early examples or light prompts ("What did you do just now that worked?") to build fluency.
\textbf{Evidence base from related literature:} This aligns with double-loop learning (Argyris and Schön), where patterns of thought and action are surfaced through inquiry. It's also reinforced in cognitive apprenticeship, which models strategies before expecting learners to apply them.

\paragraph*{{\scshape Timebox the Core}}

\noindent \textbf{Context:} You have a limited workshop time (e.g., 60–120 minutes).
\textbf{If:} You want to run through all three PLACARD components,  \textbf{But:} you risk running out of time or overwhelming participants,
\textbf{Then:} Identify and focus on a minimal viable loop (e.g. Situation → Scenario → Reflection) and simplify supporting activities.
\textbf{Evidence base from related literature:} Agile retrospectives and timeboxing practices in design sprints demonstrate how temporal constraints support clarity, focus, and iterative learning without requiring completion of all intended steps.

\paragraph*{{\scshape Role Reveal}}

\noindent \textbf{Context:} You’re running a hybrid or multi-facilitator session.
\textbf{If:} Facilitation roles are unclear,  \textbf{But:} smooth coordination is needed to support the flow,
\textbf{Then:} Make the roles explicit (e.g., {\scshape Linker}, {\sc Reflector}) and let participants know who does what.
\textbf{Evidence base from related literature:} In distributed team management and collaborative learning, named roles (e.g., Scrum Master, Scribe) are shown to enhance group performance and task clarity. This also connects to facilitation protocols in online and hybrid pedagogies.

\subsubsection*{C.~Making it Actionable}

\paragraph*{{\scshape Harvest Before You Close}}

\noindent \textbf{Context:} You’re running an engaging session.
\textbf{If:} Participants have surfaced great insights,  \textbf{But:} waiting until the end to name patterns might mean running out of time,
\textbf{Then:} Add a short mid-session or pre-wrap-up step to ask: “What’s something you did or saw that could be reused elsewhere?”
\textbf{Evidence base from related literature:} After Action Reviews (military, management), Feedback Wheels (relationship therapy), and Agile retrospectives all highlight the value of capturing learning as it emerges—not merely as a concluding gesture.

\paragraph*{{\scshape Pattern the Play}}

\noindent \textbf{Context:} You’re using creative methods (cards, music, improv) as part of your workshop.
\textbf{If:} These are treated as energizers or tangents,  \textbf{But:} you also want them to inform the core inquiry,
\textbf{Then:} Treat playful activities as valid epistemic inputs. Ask: “what design moves or values emerged here?”
\textbf{Evidence base from related literature:} Design ethnography, improv for innovation, and serious games scholarship all provide grounding for the idea that nonverbal, embodied, and affective practices contribute directly to sense-making and design thinking.

\paragraph*{{\scshape Meta-Reflection Loop}}

\noindent \textbf{Context:} You’re using a structured method like PLACARD itself.
\textbf{If:} You reflect only on the topic (e.g., music and AI), \textbf{But:} you find yourself neglecting the method,
\textbf{Then:} Add a closing question or section to reflect on the method’s affordances, limits, and what to revise.
\textbf{Evidence base from related literature:} This is consistent with reflective practice (Schön), meta-evaluation in participatory action research, and “method feedback” sessions used in critical design studies. It also draws from Buddhist and contemplative traditions of meta-awareness and iterative refinement (e.g., Dependent Origination).

\subsection*{Candidate Patterns for Pattern-competent agents}

\paragraph*{{\scshape Pattern Activation}}

\noindent \textbf{Context:} An agent receives a new task or encounters a novel situation.
 \textbf{If:} The system has access to a pattern catalogue relevant to the domain,
 \textbf{But:} Not all patterns are applicable or salient,
 \textbf{Then:} Retrieve candidate patterns using contextual filters (e.g., task type, known tensions) to prime downstream reasoning.

 \paragraph*{{\scshape Tension Detection}}

\noindent \textbf{Context:} An agent is assessing a situation, proposal, or evolving dialogue.
 \textbf{If:} Indicators suggest divergence, contradiction, or friction,
 \textbf{But:} These tensions remain unmarked and unresolved,
 \textbf{Then:} Flag the tension explicitly and propose an exploratory frame (e.g., apply CLA or consult conflict-resolution patterns) to surface underlying issues.

 \paragraph*{{\scshape Candidate Move Generation}}

\noindent \textbf{Context:} An agent has identified a situation in which one or more patterns apply.
 \textbf{If:} The context supports multiple interpretations or strategies,
 \textbf{But:} There’s uncertainty about how to instantiate the pattern concretely,
 \textbf{Then:} Generate a set of plausible candidate actions based on pattern logic, rank or cluster them using salience heuristics, and propose next steps.

 \paragraph*{{\scshape Shared Memory Architecture}}

\noindent \textbf{Context:} Multiple agents operate in a dynamic environment over time.
 \textbf{If:} Patterns are used, modified, or co-created through interaction,
 \textbf{But:} There is no mechanism for recording or retrieving this history,
 \textbf{Then:} Store pattern instances, adaptations, and usage contexts in a persistent shared memory, enabling learning and reuse.

 \paragraph*{{\scshape Feedback Rhythms}}

\noindent \textbf{Context:} A system includes agents collaborating over long timeframes or iterations.
 \textbf{If:} Patterns and strategies become outdated or misaligned,
 \textbf{But:} There is no regular cadence for reviewing or revising them,
 \textbf{Then:} Schedule periodic feedback sessions or checkpoints where agents reflect on outcomes, revise patterns, and tune system-wide epistemic strategies.

\paragraph*{{\scshape Pattern Diffusion}}

\noindent \textbf{Context:} A new pattern proves effective within a local agent subgroup.
 \textbf{If:} Other agents face similar challenges or tasks,
 \textbf{But:} There’s no pathway for knowledge transfer or generalization,
 \textbf{Then:} Broadcast the pattern across the system with metadata (e.g., domain, confidence score, context) to allow others to trial, adapt, or contest it.

\subsection*{Candidate Patterns for Pattern-competent Agent Environments}

\paragraph*{{\scshape Institutional Drift}}
\noindent \textbf{Context:} Repeated MAS sessions produce cumulative outcomes.
\textbf{If:} Pattern use becomes routinized,
\textbf{But:} Changing contexts or challenges are not reflected in pattern updates,
\textbf{Then:} Trigger a meta-level PAR to recontextualize relevance.
\textbf{Because:} Institutions require periodic re-grounding to avoid ossification.

\paragraph*{{\scshape Norm Seed}}
\noindent \textbf{Context:}  A new MAS instance is initiated.
\textbf{If:} There are no agreed-upon norms,
\textbf{But:} Some agents possess relevant precedents or analogues,
\textbf{Then:} Seed group interaction with foundational mini-patterns.
\textbf{Because:} Initial framing influences long-term behavior and coherence.

\paragraph*{{\scshape Boundary Oscillation}}
\noindent \textbf{Context:}  MAS agents engage across different domains or networks.
\textbf{If:} Norms, roles, or goals diverge significantly,
\textbf{But:} Agents have memory of shared pattern lineage,
\textbf{Then:} Engage a boundary negotiation pattern to recalibrate alignment.
\textbf{Because:} Distributed epistemic systems must periodically reconverge.

\paragraph*{{\scshape Reflection Scaffold Upgrade}}
\noindent \textbf{Context:}  Existing reflection protocols underperform in capturing learning.
\textbf{If:} Meta-cognition fails to yield new patterns,
\textbf{But:} System logs or feedback signals suggest deeper insight is possible,
\textbf{Then:} Introduce upgraded reflection scaffolds (e.g., layered PAR, narrative loops).
\textbf{Because:} Developmental learning depends on scaffolding that evolves with system maturity



\renewcommand\bibname{References}
\renewcommand\refname{References}

\bibliographystyle{ACM-Reference-Format-Journals}
\bibliography{./main.bib}


\end{document}
